\chapter{项目内容}

\section{客户端}

\subsubsection{客户端前台}
前台是java语言的显示界面
\begin{itemize}
    \item 进行网络检测并获取上联物理接口IPV6地址
    \item 启动后台线程
    \item 开启定时器刷新界面
    \item 界面显示网络状态
    \item 开启安卓VPN服务
\end{itemize}
\subsubsection{客户端后台}
后台是C语言客户端与4over6隧道服务器之间的数据交互
\begin{itemize}
    \item 连接服务器
    \item 获取下联虚接口IPV4地址并通过管道传到前台
    \item VPN程序通过打开/dev/获取前台传送到后台的虚接口描述符
    \item 读写虚接口
    \item 对数据进行解封装
    \item 通过IPV6套接字与4over6隧道服务器进行数据交互
    \item 实现保活机制,定时给服务器发送keeplive消息
\end{itemize}

\section{服务器端}
服务端在linux环境下运行,主要有下面几个功能:
\begin{itemize}
    \item 创建IPV6 TCP套接字,监听服务器和客户端之间的数据通信
    \item 维护虚接口,实现对虚接口的读写操作
    \item 维护IPV4地址池,实现为新连接客户端分配IPV4地址
    \item 维护客户信息表,保存IPV4地址与IPV6套接字之间的映射关系
    \item 读取客户端从IPV6 TCP套接字发送来的数据,实现对系统的控制消息和数据消息的处理
    \item 实现对数据消息的解封装,并写入虚接口
    \item 实现对虚接口接收到的数据报文进行封装,通过IPV6套接字发送给客户端
    \item 实现保活机制,监测客户端是否在线,并且定时给客户端发送keeplive消息
\end{itemize}