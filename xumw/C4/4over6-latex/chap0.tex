\chapter{项目简介}


\section{项目概述}
% 本条应简述本文档适用的系统和软件的用途。它应描述系统与软件的一般性质;概述系统开发、运行和维护的历史;标识项目的投资方、需方、用户、开发方和支持机构;标识当前和计划的运行现场;并列出其他有关文档。

IPV4 over IPV6,简称“4over6”是IPV4向IPV6发展进程中,向纯IPV6主干网过渡提出的一种新技术,可以最大程度地继承基于IPV4网络和应用,实现IPV4向IPV6平滑的过渡。
该项目通过实现IPV4 over IPV6隧道最小原型验证系统,以实现在安卓手机上,通过IPv6网络访问IPv4网站的功能。 


% \section{项目目的}
% % 本条应包含本文档适用的系统和软件的完整标识,(若适用)包括标识号、标题、缩略词语、版本号、发行号。

% \begin{itemize}
%     \item 掌握Android下应用程序开发环境的搭建和使用
%     \item 掌握IPv4 over IPv6隧道的工作原理
% \end{itemize}

\section{客户端项目概述}
% 本条应概括本文档的用途与内容,并描述与其使用有关的保密性与私密性要求。
在安卓设备上实现一个4over6隧道系统的客户端程序,内容如下:

\begin{itemize}
    \item 实现安卓界面程序,显示隧道报文收发状态(java语言)
    \item 启用安卓VPN服务(java语言)
    \item 实现底层通信程序,对4over6隧道系统控制消息和数据消息的处理(C语言)
\end{itemize}

\section{服务器项目概述}

在linux系统下,实现4over6隧道系统服务端程序,内容如下:
\begin{itemize}
    \item 实现服务端与客户端之间控制通道的建立与维护
    \item 实现对客户端网络接口的配置
    \item 实现对4over6隧道系统数据报文的封装和解封装
\end{itemize}