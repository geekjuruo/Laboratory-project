\chapter{实验心得与体会}

\section{柳瑞阳同学的心得体会}
在本次课程中,虽然只是有两周的教学讲解,其余时间都大实验时间。但是整体收获依旧很大。需要学习的东西很多。我主要负责4over6客户端的搭建。从一开始的Android Studio环境搭建就费尽不少波折。再到学习Java 内嵌 C 语法,前后台虚拟管道的连接等。虽然这些只是实验的前期准备工作,但是也占用了实验相当大的一部分时间。在这里我要给实验指导书点一个赞,实验指导书内容详尽,为我们提供了不少的帮助。虽然实现流程按照实验指导书编写,但是自己也对于这个流程更加熟悉。助教曾问我:“如果实验指导书不给你们写出来实验实现流程,你们自己能不能实现?”我说,“很玄”。实验指导书为我们实验具体实施提出了明路。但与此同时也有很多坑,在实验指导书中未说明。例如服务器会像客户端多次返回101地址请求包,如果客户端对于每个包都处理,反而会造成不可预见的错误。这个问题的发现也消耗了我们一部分时间。另外实验指导书服务器和客户端部分对于包长度length的定义不够统一,希望明年可以改善一下。最后就是实验室的Wifi环境不尽人意,需要麻烦助教到宿舍进行检查。希望明年有所改善。最后因为我们一开始以为服务器和客户端是解耦合的,独立开发,客户端依托于实验室服务器进行测试。但是最后才发现,也需要和自己的服务器的同学多交流沟通,以确定部分操作的统一(例如包处理)。为此我们在客户端验收结束后还进行了一段时间的和自己的服务器的适配工作。这段时间内感谢队友的帮助,其他组的帮助,以及助教的答疑,老师的讲解与鼓励。网络编程实验收获到的,远不止如何4over6。

\section{李映辉同学的心得体会}
在此次实验中,通过这样一个有意义的4over6大作业,的确学到了很多。我主要负责客户端和服务器端的代码调试以及文档撰写。在本次实验过程中,我们遇到了很多问题,也走了很多弯路,但所幸我有两位靠谱的队友,在最终成功调试出了可以成功使用IPv4 over IPv6的安卓客户端和服务器。这其中走过的弯路包括“Android Studio环境版本配置的问题”、“客户端服务器端连通出现问题“、“客户端服务器互相发消息无法识别”等。也正是因为在这一次次的错误能够重新站起来,我们每一个队员才从中获得了很多,学到了很多,每一次失败和错误都是为最后的成功做铺垫,每一次失败和错误都是非常宝贵的经验。总的来说,通过这次实验,我再一次丰富了Android Studio开发的经验,理解并掌握了面向Android的4over6隧道原理,受益匪浅,收获颇丰。除此之外,对于原理课上所讲的一些仅仅停留在书本文字层面的知识内容我也有了更深的理解和体会,并通过自己的切身实践深化了对于转发、隧道等概念的理解,这也让我真实地感受到了什么叫“实践出真知”。最后,在此向对我们提供热心帮助和指导的助教表示真挚的感谢和敬意。

\section{曾军同学的心得体会}
本次网络专题训练我负责的是4over6服务器的搭建工作,这也是我第一次在linux下进行服务器开发工作,在开始实现服务器之前我阅读了大量资料,也对linux下基于socket的通信有了更加深刻的认识。但是在实现服务器的时候仍然遇到了很多的问题:实验指导书上的select模型针对新版本的linux存在各种不兼容问题和效率上的问题,在使用select模型实现服务器之后经常出现服务器空闲但仍然无法处理数据包的情况,调试了很久也没有解决这个问题。在尝试了很多办法之后,我最终借鉴往届学长的方法使用新兴的epoll框架实现4over6服务器,epoll服务器没有等待队列的概念,他把socket固化成为一个个句柄,并且使用类似于信号槽的机制管理句柄,实现高效无阻塞的多路复用。本次实验让我对ipv4和ipv6有了更深刻的认识,这是对大三上学期计算机网络原理的巩固和训练。通过在服务器中使用socket编程,我的网络编程技术也有了很大的进步,本次实验我也认识到不能把自己的视野放在成熟的模型上,要勇于尝试新的事物,新兴的事物往往会带来意想不到的惊喜,我在今后的学习研究中也会继续努力,争取加深自己对计算机网络的认识,最后谢谢老师和助教的指导!