\chapter{项目中遇到的问题}

\section{客户端遇到的问题}
我们客户端的开发在项目过程中发现一个问题,即服务器可能向客户端发送不止一个 IP 地址回应包,并且包内容相同。然而在项目中我们仅仅想对 IP 地址回应包进行唯一的一次处理,因为涉及新线程的开启等工作。故我们对于 IP 地址回应包是否进行处理除了判断其数据包类型外,还需要判断是否以前收到了 IP 地址回应包,如果收到了,则将此包丢弃。如果以前没有收到,则处理此包。
\section{服务器端遇到的问题}
服务器端开发我们遇到的最大的问题就是,一开始我们天真的以为客户端和服务器端的开发是完全解耦合的,于是负责客户端开发的同学和负责服务器端开发的同学一开始就是各自为战。但当客户端和服务器端真正一起跑起来之后,才发现并不是像我们想象的那么简单,最大的问题还是服务器端和客户端之间互相通信的问题,服务器能够接收到客户端发来的包,但却无法识别。也就是说服务器端和客户端的包的定义不一致,这也就是说服务器端和客户端并不是完全解耦合的。这一bug的调试极大程度地拖慢了我们的进度,当然最终我们还是成功实现了我们自己开发的服务器端和客户端的通信,并可以成功使用我们的客户端通过我们的服务器进行IPv4 over IPv6的操作。